\documentclass[UTF8]{ctexart}

\usepackage{listings}
\usepackage[dvipsnames]{xcolor}

% 在导言区进行样式设置
\lstset{
    language=Python, % 设置语言
    basicstyle=\ttfamily, % 设置字体族
    breaklines=true, % 自动换行
    keywordstyle=\bfseries\color{NavyBlue}, % 设置关键字为粗体,颜色为 NavyBlue
    morekeywords={}, % 设置更多的关键字,用逗号分隔
    emph={self}, % 指定强调词,如果有多个,用逗号隔开
    emphstyle=\bfseries\color{Rhodamine}, % 强调词样式设置
    commentstyle=\itshape\color{black!50!white}, % 设置注释样式,斜体,浅灰色
    stringstyle=\bfseries\color{PineGreen!90!black}, % 设置字符串样式
    columns=flexible, %代码紧凑一些
    numbers=left, % 显示行号在左边
    numbersep=2em, % 设置行号的具体位置
    numberstyle=\footnotesize, % 缩小行号
    frame=single, % 边框
    framesep=1em % 设置代码与边框的距离
}

\begin{document}

\begin{lstlisting}
import random
import collections
Card = collections.namedtuple('Card', ['rank', 'suit'])
# 一个叫做 FrenchDesk 的类
class FrenchDesk:
    ranks = [str(n) for n in range(2, 11)] + list('JQKA')
    suits = 'spades diamonds clubs hearts'.split()
    
    def __init__(self):
        self._cards = [Card(rank, suit) for rank in self.ranks for suit in self.suits]
        
    def __len__(self):
        return len(self._cards)
        
    def __getitem__(self, position):
        return self._cards[position]
desk = FrenchDesk()
\end{lstlisting}

\end{document}