%%%%%%%%%%%%%%%%%%%%%%%%%%%%%%%%%%%%%%%%%
% Professional Table
% LaTeX Template
% Version 1.1 (19/09/2018)
%
% This template was downloaded from:
% http://www.LaTeXTemplates.com
%
% Author:
% Vel (vel@latextemplates.com) 
%
% License:
% CC BY-NC-SA 3.0 (http://creativecommons.org/licenses/by-nc-sa/3.0/)
%
% Note: to use this table in another LaTeX document, you will need to copy
% the \usepackage{booktabs} line to the new document and paste it before 
% \begin{document}. The table itself can then be pasted anywhere in the new
% document.
%
%%%%%%%%%%%%%%%%%%%%%%%%%%%%%%%%%%%%%%%%%

\documentclass[a4paper]{article}

\usepackage{booktabs} % Required for better horizontal rules in tables

\usepackage{fouriernc} % Use the New Century Schoolbook font

\usepackage{multirow}

\usepackage[font=small, labelsep=space, labelformat=default]{caption}

\captionsetup[table]{ name=Tab. }

\begin{document}

\begin{table} % Add the following just after the closing bracket on this line to specify a position for the table on the page: [h], [t], [b] or [p] - these mean: here, top, bottom and on a separate page, respectively
\centering % Centres the table on the page, comment out to left-justify
\begin{tabular}{l c c c c c} % The final bracket specifies the number of columns in the table along with left and right borders which are specified using vertical pipes (|); each column can be left, right or center-justified using l, r or c. Columns will widen to hold the content in them by default, to specify a precise width, use p{width}, e.g. p{5cm}
\toprule % Top horizontal line
\multirow{2}{*}{\textbf{Strain}} & \multicolumn{5}{c}{\textbf{Growth Media}} \\
% Amalgamating several columns into one cell is done using the \multicolumn command with the number of columns to amalgamate as the first argument and then the justification (l, r or c)
\cmidrule(l){2-6} % Horizontal line spanning less than the full width of the table - you can add (r) or (l) just before the opening curly bracket to shorten the rule on the left or right side
& 1 & 2 & 3 & 4 & 5\\ % Column names row
\midrule % In-table horizontal line
GDS1002& 0.9 & 0.821 & 0.356 & 0.682 & 0.801\\ % Content row 1
NWN652 & 0.981 & 0.891 & 0.527 & 0.574 & 0.984\\ % Content row 2
PPD234 & 0.915 & 0.936 & 0.491 & 0.276 & 0.965\\ % Content row 3
JSB126 & 0.828 & 0.827 & 0.528 & 0.518 & 0.926\\ % Content row 4
JSB724 & 0.916 & 0.933 & 0.482 & 0.644 & 0.937\\ % Content row 5
\midrule % In-table horizontal line
\midrule % In-table horizontal line
Average Rate & 0.920 & 0.882 & 0.477 & 0.539 & 0.923\\ % Summary/total row
\bottomrule % Bottom horizontal line
\end{tabular}
\caption{Table caption text} % Table caption, can be commented out if no caption is required
\label{tab:template} % A label for referencing this table elsewhere, references are used in text as \ref{label}
\end{table}

A reference to Table \ref{tab:template}.

\end{document}